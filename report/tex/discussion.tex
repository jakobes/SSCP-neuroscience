\section{Discussion - Signaling}
Our signalling models show a gain-of-function caused by mutation one (Q1489K), seen by the higher
firing rates elicited by a step-wise increase on current stimulation as the fraction of mutated
channels increases. In contrast, mutation two (L1649Q) demonstrates a loss-of-function as the
fraction of mutated channels increases.

Therefore, we conclude that at a critical level of mutated sodium channels, we get a increase in
GluR levels which in turn would cause feedback by affecting the neurons response to future action
potentials and synaptic communication.

Interestingly, these relationship could be the opposite depending on the cellular
population affected. For example, epileptic seizures may be triggered by a loss of function of
inhibitory neurons, which may result in a gain of function of excitatory cortical neurons.
