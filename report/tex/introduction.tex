\section{Introduction}

Imagine a neuron on a stage playing a guitar, with microphone that is a bit too sensitive. Picking
up the  speaker, it produces an awful sound. It is the same that happens during an epileptic
seizure. Mutations can cause a network of neurons are a bit too sensitive to stimulation,
and end up creating a positive feed back loop, stimulating each other, causing an epileptic seizure.
In 2015, 39.2 million people suffered from epilepsy worldwide, making it one of the most common
ailments \cite{vos2016global}. Therefore, the study of epilepsy, and its associated mutations, is
of great importance.

A correlation has been shown to exist between several genetic variants affecting ion channels in
neurons and instances of diseases such as familial migraine, epilepsy and schizophrenia
\cite{vanmolkot2007novel,escayg2010sodium,maki2016functional}. By a suitable change of parameters in
the compartmental pyramidal layer 5 cell model (L5PC) \cite{hay2011models}, altered neuron spiking
due to two SCN1A mutations was characterized separately. Affected and unaffected networks were
subsequently simulated in order to ascertain the effect of the mutation. Parameters found by a fit
to the single neuron frequency-current (f-I) curve were used to encode the presence of mutated
channels in the network setting.
