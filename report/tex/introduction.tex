\section{Introduction}

%% Introduce epilepsy
Imagine a neuron on a stage playing a guitar, with microphone that is too sensitive. Picking
up the  speaker, it produces an awful feedback sound. It is the same that happens during an
epileptic seizure. Epilepsy is a neurological condition predominantly characterized by the
protracted recurrence of seizures. In 2015, 39.2 million people suffered from epilepsy worldwide,
making it one of the most common ailments \cite{vos2016global}. Therefore, further research must be
conducted to understand the mechanism through which epileptic seizures manifest for the future
development of targeted therapeutics.

% Talk about mutations
One avenue of research has focused on the SCNA1 gene, a sodium channel gene
with a history of mutations associated with epilepsy. For instance, mutants exhibit subtle
differences in voltage-gated sodium ion channel density along the cellular membrane. These changes
in channel density have been thought to underlie the physiological disturbance that results in
abnormal neurological functioning found in epilepsy. Here, we utilised two previously developed
models of SCNA1 mutations, the Q1489K and L1649Q variants, to gain a greater understanding of the
specific downstream cellular signalling mechanisms that are affected by changing the density of
mutated membrane sodium channels.

% A short summary of methods
By a suitable change of parameters in the compartmental pyramidal layer 5 cell model (L5PC)
\cite{hay2011models}, altered neuron spiking due to two SCN1A mutations was characterised
separately. Affected and unaffected networks were subsequently simulated in order to ascertain the
effect of the mutation. Parameters found by a fit to the single neuron frequency-current (f-I) curve
were used to encode the presence of mutated channels in the network setting. Neuron network dynamics
were investigated by a network model of excitatory and inhibitory neurons, using the
parameter-fitted integrate and fire model for computational efficiency.

% Mutations can cause a network of neurons are a bit too sensitive to stimulation,
% and end up creating a positive feed back loop, stimulating each other, causing an epileptic seizure.

% A correlation has been shown to exist between several genetic variants affecting ion channels in
% neurons and instances of diseases such as familial migraine, epilepsy and schizophrenia
% \cite{vanmolkot2007novel,escayg2010sodium,maki2016functional}. 
