\section{Introduction}

Imagine a neuron on a stage playing a guitar, with microphone that is a bit too sensitive. Picking
up the  speaker, it produces an awful sound. It is the same that happens during an epileptic
seizure. Mutations can cause a network of neurons are a bit too sensitive to stimulation,
and end up creating a positive feed back loop, stimulating each other, causing an epileptic seizure.
In 2015, 39.2 million people suffered from epilepsy worldwide, making it one of the most common
ailments \cite{vos2016global}. Therefore, the study of epilepsy, and its associated mutations, is
of great importance.

The mutations in the gene SGNA1 has a history of association with epilepsy, making it an obvious
target for further investigation. Even subtle differences in sodium channel function can cause
abnormal neuron behaviour \textcolor{red}{citation needed}.
